\documentclass[man]{apa6}
\usepackage{lmodern}
\usepackage{amssymb,amsmath}
\usepackage{ifxetex,ifluatex}
\usepackage{fixltx2e} % provides \textsubscript
\ifnum 0\ifxetex 1\fi\ifluatex 1\fi=0 % if pdftex
  \usepackage[T1]{fontenc}
  \usepackage[utf8]{inputenc}
\else % if luatex or xelatex
  \ifxetex
    \usepackage{mathspec}
  \else
    \usepackage{fontspec}
  \fi
  \defaultfontfeatures{Ligatures=TeX,Scale=MatchLowercase}
\fi
% use upquote if available, for straight quotes in verbatim environments
\IfFileExists{upquote.sty}{\usepackage{upquote}}{}
% use microtype if available
\IfFileExists{microtype.sty}{%
\usepackage{microtype}
\UseMicrotypeSet[protrusion]{basicmath} % disable protrusion for tt fonts
}{}
\usepackage{hyperref}
\hypersetup{unicode=true,
            pdftitle={A comparison of meta-analysis, mega-analysis, and a hybrid approach},
            pdfauthor={Ezequiel Koile, Sho Tsuji, \& Alejandrina Cristia},
            pdfborder={0 0 0},
            breaklinks=true}
\urlstyle{same}  % don't use monospace font for urls
\usepackage{graphicx,grffile}
\makeatletter
\def\maxwidth{\ifdim\Gin@nat@width>\linewidth\linewidth\else\Gin@nat@width\fi}
\def\maxheight{\ifdim\Gin@nat@height>\textheight\textheight\else\Gin@nat@height\fi}
\makeatother
% Scale images if necessary, so that they will not overflow the page
% margins by default, and it is still possible to overwrite the defaults
% using explicit options in \includegraphics[width, height, ...]{}
\setkeys{Gin}{width=\maxwidth,height=\maxheight,keepaspectratio}
\IfFileExists{parskip.sty}{%
\usepackage{parskip}
}{% else
\setlength{\parindent}{0pt}
\setlength{\parskip}{6pt plus 2pt minus 1pt}
}
\setlength{\emergencystretch}{3em}  % prevent overfull lines
\providecommand{\tightlist}{%
  \setlength{\itemsep}{0pt}\setlength{\parskip}{0pt}}
\setcounter{secnumdepth}{0}
% Redefines (sub)paragraphs to behave more like sections
\ifx\paragraph\undefined\else
\let\oldparagraph\paragraph
\renewcommand{\paragraph}[1]{\oldparagraph{#1}\mbox{}}
\fi
\ifx\subparagraph\undefined\else
\let\oldsubparagraph\subparagraph
\renewcommand{\subparagraph}[1]{\oldsubparagraph{#1}\mbox{}}
\fi

%%% Use protect on footnotes to avoid problems with footnotes in titles
\let\rmarkdownfootnote\footnote%
\def\footnote{\protect\rmarkdownfootnote}


  \title{A comparison of meta-analysis, mega-analysis, and a hybrid approach}
    \author{Ezequiel Koile\textsuperscript{a}, Sho Tsuji\textsuperscript{b}, \&
Alejandrina Cristia\textsuperscript{c}}
    \date{}
  
\shorttitle{individual variation in infant speech processing}
\affiliation{
\vspace{0.5cm}
\textsuperscript{a} ADD\\\textsuperscript{b} ADD\\\textsuperscript{c} ADD}
\usepackage{csquotes}
\usepackage{upgreek}
\captionsetup{font=singlespacing,justification=justified}

\usepackage{longtable}
\usepackage{lscape}
\usepackage{multirow}
\usepackage{tabularx}
\usepackage[flushleft]{threeparttable}
\usepackage{threeparttablex}

\newenvironment{lltable}{\begin{landscape}\begin{center}\begin{ThreePartTable}}{\end{ThreePartTable}\end{center}\end{landscape}}

\makeatletter
\newcommand\LastLTentrywidth{1em}
\newlength\longtablewidth
\setlength{\longtablewidth}{1in}
\newcommand{\getlongtablewidth}{\begingroup \ifcsname LT@\roman{LT@tables}\endcsname \global\longtablewidth=0pt \renewcommand{\LT@entry}[2]{\global\advance\longtablewidth by ##2\relax\gdef\LastLTentrywidth{##2}}\@nameuse{LT@\roman{LT@tables}} \fi \endgroup}


\DeclareDelayedFloatFlavor{ThreePartTable}{table}
\DeclareDelayedFloatFlavor{lltable}{table}
\DeclareDelayedFloatFlavor*{longtable}{table}
\makeatletter
\renewcommand{\efloat@iwrite}[1]{\immediate\expandafter\protected@write\csname efloat@post#1\endcsname{}}
\makeatother

\authornote{

Correspondence concerning this article should be addressed to Ezequiel
Koile, ADD. E-mail: ADD}

\abstract{
Laboratory measures of infant speech perception have been central to the
development of theories of infant language acquisition, and could be
valuable predictors of important individual and group variation. A
recent report suggests that these measures' psychometric properties may
be limited, based on a meta-analytic analysis. We re-analyze those data
using a mega-analytic approach, as well as a variety of hybrid
approaches. We find that (a) the results of meta- and mega-analyses
diverge significantly, and (b) a mega-analytic approach can be more
powerful in detecting stability in performance across days. However,
since it is often difficult to recover original data, we also explore a
hybrid approach, in which some studies are represented by group
statistics, and others by the original data, assessing to what extent
biased data sharing may impact overall conclusions.


}

\begin{document}
\maketitle

\subsection{Rough paper outline}\label{rough-paper-outline}

One paragraph per bullet point

\begin{itemize}
\item
  infant speech perception measures have been central to the development
  of theories of infant language acquisition, eg phonology \& lexicon
  start developing before 1 year of age
\item
  infant speech perception measures could be valuable predictors of
  important individual and group variation, eg correlations these \&
  vocabulary or comparisons between groups at risk or not (Cristia et al
  2013)
\item
  what are the psychometric properties? Psychometry is crucial -- eg
  test-retest reliability gives upper bound on meaningful variance:
  \enquote{the maximum validity of any measure is the square root of its
  reliability}
\item
  only two studies published on test-retest of infant speech perception
  measures, \& second contains data on first. Cristia et al use
  meta-analytic method and find, across 12 studies, weighted median r of
  zero
\item
  this means any correlation/intervention work using these measures is
  suspect because they have virtually no good reliability
\item
  correlations within each study make sense, but do not capture same
  information as correlations collapsing across studies; or considering
  studies as structured
\item
  current trends in genetics \& brain studies pushing for mega-, over
  meta-, analyses because structured sources of variance can be better
  accounted for, and analyses may have more power
\item
  here, we reanalyze Cristia's data to revisit the question of
  reliability, and ask
\item
  in mega-analysis, do you also find basically no prediction of test2
  from test1?
\item
  how should structure be accounted for - are studies all different from
  each other?
\item
  what happens if you only have some data from some studies -- picked at
  random? (assuming original authors do not withhold the data for any
  reason that is related to the data itself)
\item
  and if you only have data from large studies? (authors who ran more
  babies are more motivated to share)
\item
  and if you only have data from studies with large main effects?
  (defined as the average between effect at test1 and effect at test2 --
  intuition is that authors with strong effects believe their data more)
\item
  and if you only have data from studies with large test-retest
  correlations? (idea: authors who find reliability more likely to share
  raw data)
\end{itemize}

\subsection{Methods}\label{methods}

Very short because we refer to previous paper for full description of
experiments

table of experiments: short names, short description, N of children,
mean age

We got data from osf, using R, this paper uses Rmd in RStudio \& papaja
for increased reproducibility.

\subsection{Results}\label{results}

\begin{itemize}
\tightlist
\item
  how should structure be accounted for - are studies all different from
  each other?
\end{itemize}

explain use of AIC to compare models, using also conceptual reasons to
group studies -- ending up with 5 clusters

\begin{itemize}
\tightlist
\item
  in mega-analysis, do you also find basically no prediction of test2
  from test1?
\end{itemize}

no, we get something pretty different. Explain why

\begin{itemize}
\tightlist
\item
  use a graph to represent all of the hybrid results (max 4k words!)
\end{itemize}

\subsection{Discussion}\label{discussion}

\begin{itemize}
\tightlist
\item
  under what conditions can we trust infant speech perception measures
  of individual variation?
\item
  we recommend mega- over meta-analysis
\item
  explain under what conditions this holds, and when mega-analysis
  provides biased view of data
\end{itemize}

\subsection{References}\label{references}

\setlength{\parindent}{-0.5in} \setlength{\leftskip}{0.5in}


\end{document}
